
\documentclass[a4paper,12pt]{article} % тип документа

% report, book

%  Русский язык

\usepackage[T2A]{fontenc}			% кодировка
\usepackage[utf8]{inputenc}			% кодировка исходного текста
\usepackage[english,russian]{babel}	% локализация и переносы


% Математика
\usepackage{amsmath,amsfonts,amssymb,amsthm,mathtools}

%форматирование
\usepackage{float}
\usepackage{geometry}
%Рисунки
\usepackage{graphicx}
\usepackage{wrapfig}

\usepackage{hyperref}
\usepackage[rgb]{xcolor}
\hypersetup{
	colorlinks = true,
	urlcolor = blue
}

\usepackage{wasysym}

%Заговолок
\author{Колобаев Дмитрий\\Группа Б01-903}
\title{Лабораторная работа №1 \\ОДУ} % Номер и название работы


\begin{document}
\maketitle

\begin{flushleft}
	\textbf{Задание VIII.11.2}: \\
	Изучите поведение численного решения ОДУ второго порядка (уравнения Ван-дер-Поля):
	$$y'' + e(y^2 - 1)y' + y = 0$$
	Представленного в виде системы двух ОДУ первого порядка
	$$
	\begin{cases}
		x' = z \\
		z' = e(1 - x^2)z - x, e>0
	\end{cases}
	$$

	в зависимости от изменения параметра $e$ $(0,01 < e < 100)$. Использовать методы Рунге–Кутты порядка 1,2,3,4 и методы Адамса порядка 2, 3, 4.
\end{flushleft}

\section*{Метод Рунге-Кутты}

S-стадийный одношаговый метод решения задачи Коши для ОДУ.

\begin{eqnarray*}
	k_1 & = & f(t_n, y_n),\\
	... && \\
	k_s & = & f(t_n + c_s\tau, y_n + \tau(\sum_{i=1}^{s-1}a_{si}k_i) ),\\
	y_{n+1} & = & y_n + \tau(\sum_{j=1}^sb_jk_j)
\end{eqnarray*}

Коэффициенты в методе Рунге-Кутты задаются таблицей Бутчера.

\begin{tabular}[H]{|c|c|}
	\hline
	$\vec{c}$ & [$a_{ij}$] \\ \hline
	& $b^t$ \\
	\hline
\end{tabular}

Использованые в работе варианты метода:

\begin{itemize}
	\item Метод Эйлера 1-го порядка \\
	\begin{tabular}[H]{|c|c|}
		\hline
		0 & 0 \\ \hline
		& 1 \\
		\hline
	\end{tabular}

	\item Метод Эйлера с пересчетом 2-го порядка\\
	\begin{tabular}[H]{|c|c|c|}
		\hline
		0 & 0 & 0\\ \hline
		1 & 1 & 0\\ \hline
		& 1/2 & 1/2 \\
		\hline
	\end{tabular}

	\item Метод Хойна 3-го порядка\\
	\begin{tabular}[H]{|c|c|c|c|}
		\hline
		0 & 0 & 0 & 0\\ \hline
		1/3 & 1/3 & 0 & 0\\ \hline
		2/3 & 0 & 2/3 & 0 \\ \hline
		& 1/4 & 0 & 3/4 \\
		\hline
	\end{tabular}

	\item Метод Рунге-Кутты 4-го порядка \\
	\begin{tabular}[H]{|c|c|c|c|c|}
		\hline
		0 & 0 & 0 & 0 & 0\\ \hline
		1/2 & 1/2 & 0 & 0 & 0\\ \hline
		1/2 & 0 & 1/2 & 0 & 0 \\ \hline
		1 & 0 & 0 & 1 & 0 \\ \hline
		& 1/6 & 2/6 & 2/6 & 1/6 \\
		\hline
	\end{tabular}
\end{itemize}

\section*{Методы Адамса}
Многошаговый одностайдийный метод решения ОДУ. Явные методы Адамса, от первого до четвертого порядка аппроксимации, на равномерной сетке представляются в виде

\begin{eqnarray*}
	y_{n+1} & = & y_n + \tau f_n\\
	y_{n+1} & = & y_n + \tau (\frac{3}{2}f_n - \frac{1}{2}f_{n-1})\\
	y_{n+1} & = & y_n + \tau (\frac{23}{12}f_n - \frac{16}{12}f_{n-1} + \frac{5}{12}f_{n-2})\\
	y_{n+1} & = & y_n + \tau (\frac{55}{24}f_n - \frac{59}{24}f_{n-1} + \frac{37}{24}f_{n-2} - \frac{9}{24}f_{n-3})
\end{eqnarray*}

\subsection*{Фазовые траектории}

В фазовых траекториях наблюдаются предельные циклы осциллятора. При малых $e$ траектории похожи на гармонический осциллятор, с ростом параметра максимцмы эллипса сдвигаются от центра в первую и третью четверти. С ростом параметра также увеличивается период одного колебания Поэтому если период становится больше чем $t_{max} = 100$ отсчетов по времени, то цикл оказывается неполным.

\newcommand{\kuttaGraph}[1]{
\begin{figure}[H]
	\centering
	\includegraphics[width=0.7\textwidth]{rk#1.png}
	\caption{Метод Рунге-Кутты #1-го порядка}
\end{figure}}

\kuttaGraph{1}

\kuttaGraph{2}

\kuttaGraph{3}

\kuttaGraph{4}

\newcommand{\adamsGraph}[1]{
\begin{figure}[H]
	\centering
	\includegraphics[width=0.7\textwidth]{ad#1.png}
	\caption{Метод Адамса #1-го порядка}
\end{figure}}

\adamsGraph{2}

\adamsGraph{3}

\adamsGraph{4}

\subsection*{Зависимость от времени}

Построим графики зависимости решения от времени. Они представляют собой периодические колебания. При малых значениях параметра, осциллятор похож на гармонический, при $e=0$ полуаем уравнение гармонического осциллятора в чистом виде. С ростом параметра $e$   степень затухания растет и колебания все больше не похожи на гармонические.

\begin{figure}[H]
	\centering
	\includegraphics[width=\textwidth]{time_rk4.png}
	\caption{Метод Рунге-Кутты 4-го порядка}
\end{figure}

\begin{figure}[H]
	\centering
	\includegraphics[width=\textwidth]{time_ad4.png}
	\caption{Метод Адамса 4-го порядка}
\end{figure}

\section*{Код}

Реализацию методов можно найти в папке Computational-Math:
\begin{itemize}
	\item[Рунге-Кутта - ] RKS.cpp и RKS.hpp
	\item[Адамс - ] Adams.cpp и Adams.hpp
\end{itemize}

\end{document}