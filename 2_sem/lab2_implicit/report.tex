
\documentclass[a4paper,12pt]{article} % тип документа

% report, book

%  Русский язык

\usepackage[T2A]{fontenc}			% кодировка
\usepackage[utf8]{inputenc}			% кодировка исходного текста
\usepackage[english,russian]{babel}	% локализация и переносы


% Математика
\usepackage{amsmath,amsfonts,amssymb,amsthm,mathtools}

%форматирование
\usepackage{float}
\usepackage{geometry}
%Рисунки
\usepackage{graphicx}
\usepackage{wrapfig}

\usepackage{hyperref}
\usepackage[rgb]{xcolor}
\hypersetup{
	colorlinks = true,
	urlcolor = blue
}

\usepackage{wasysym}

%Заговолок
\author{Колобаев Дмитрий\\Группа Б01-903}
\title{Лабораторная работа №1 \\ОДУ} % Номер и название работы
\date{}

\begin{document}
\maketitle

\begin{flushleft}
	\textbf{Задание X.9.9}: \\
	Модель химической реакции из, получившая свое название Е5 в более ранних публикациях:
	\begin{align*}
		\dot{y_1} = & -Ay_1 - By_1y_3, \\
		\dot{y_2} = & Ay_1 - MCy_2y_3, \\
		\dot{y_3} = & Ay_1 - By_1y_3 - MCy_2y_3 + Cy_4, \\
		\dot{y_4} = & By_1y_3 - Cy_4
	\end{align*}
\end{flushleft}

Начальные условия: $y_1(0) = 1,76\cdot 10^{-3}$, все остальные переменные равны 0. Значения коэффициентов модели следующие: $A=7,89\cdot 10{-10}$, $B=1,1\cdot 10^{7}$, $C=1,13\cdot 10^{3}$, $M = 10^6$. Первоначально задача ставилась на отрезке $T_k = 1000$, но впоследствии было обнаружено, что она обладает нетривиальными свойствами вплоть до времени $T_k=10^{13}$.

\section*{Метод решения}

Воспользуемся одноитерационным методом Розенброка третьего порядка точности. Он имеет следующий вид (Петров, Лобанов "Лекции по вычислительной математике").

\begin{equation}
	(\textbf{E} -a\tau \textbf{J} - b\tau^2\textbf{J}^2)\frac{\textbf{u}_{n+1} - \textbf{u}_n}{\tau} = f[\textbf{u}_n + c\tau \textbf{f}(\textbf{u}_n)]
	\label{eq:rosenbrok}
\end{equation}

Здесь $J = \frac{\partial f}{\partial u}(u_n)$ - матрица якоби. Параметры подобраны так, чтобы обеспечить 3 порядок точности:
$$a = 1,077; b = -0,372; c = -0,577$$

Выражая $y_{n+1}$ из формулы \ref{eq:rosenbrok} получаем значение на следующей стадии.

\section*{Результаты}

Результаты вычислений представлены на рисунках ниже. Стоит отметить, что концентрации всех реагентов остаются маленькими. Однако если 2, 3 и 4 растут, то концентрация реагента 1 очень медленно уменьшается со временем.

\section*{Программная реализация}

Имплементацию использованного метода Розенброка можно найти в файлах Rosenbrok.cpp и Rosenbrok.hpp

\newcommand{\plot}[1]{
\begin{figure}[H]
	\includegraphics[width=\textwidth]{y#1.png}
\end{figure}
}

\plot{1}
\plot{2}
\plot{3}
\plot{4}



\end{document}